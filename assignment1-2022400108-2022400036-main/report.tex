\documentclass[a4paper,12pt]{article}
\usepackage[a4paper,margin=1in]{geometry}
\usepackage{graphicx}
\usepackage{listings}
\usepackage{xcolor}
\usepackage{hyperref}

\lstdefinestyle{CStyle}{
    language=C,
    basicstyle=\ttfamily\footnotesize,
    keywordstyle=\color{blue},
    commentstyle=\color{green},
    stringstyle=\color{red},
    numbers=left,
    numberstyle=\tiny,
    stepnumber=1,
    breaklines=true,
    frame=single
}

\title{Project Report  \\ \large Systems Programming – Spring 2025}
\author{Akif Yıldırım, 2022400108 \\ Oğuz Semih Arık, 2022400036 
%\\ Department of Computer Engineering \\ University Name
}
\date{\today}

\begin{document}

\maketitle

\section{Introduction}
The Witcher Tracker project simulates a text-based assistant for Geralt of Rivia. The system manages ingredients, potions, trophies, and monster knowledge using linked list structures. The main objective is to provide a dynamic, modular command-based program capable of parsing inputs and updating the internal state accordingly.

\section{Problem Description}
Geralt can perform several actions:
\begin{itemize}
    \item Loot ingredients.
    \item Learn new potion formulas or monster weaknesses.
    \item Brew potions using ingredients.
    \item Trade trophies for ingredients.
    \item Defeat monsters if prepared.
    \item Query total ingredients, potions, or trophies.
\end{itemize}

Constraints:
\begin{itemize}
    
    \item All lists must be implemented using custom linked lists.
\end{itemize}

\section{Methodology}
The solution is based on string parsing techniques in C using \texttt{strtok}, \texttt{sscanf}, and \texttt{strstr}. The project uses multiple handler functions for different commands. Dynamic memory management is handled via \texttt{malloc} and \texttt{free}.





\section{Implementation}

\subsection{Code Structure}
\begin{itemize}
    \item \texttt{main.c} — Contains all data structures, input parsing, and command handlers.
    \item \texttt{Makefile} — Compiles the program.
    
\end{itemize}

\subsection{Key Functions}
\begin{itemize}
    \item \texttt{handleLoot()} — Adds looted ingredients.
    \item \texttt{handleLearn()} — Learns formulas or monster knowledge.
    \item \texttt{handleTrade()} — Trades trophies for ingredients.
    \item \texttt{handleBrew()} — Brews potions using ingredients.
    \item \texttt{handleQuestions()} — Handles all queries.
\end{itemize}

\subsection{Sample Code}
\begin{lstlisting}[style=CStyle]
void add(Item** head, const char* name, int quantity) {
    Item* item = find(*head, name);
    if (item) item->quantity += quantity;
    else {
        Item* newItem = malloc(sizeof(Item));
        strcpy(newItem->name, name);
        newItem->quantity = quantity;
        newItem->next = *head;
        *head = newItem;
    }
    
}

\end{lstlisting}

\subsection{Challenges and Solutions}

During the implementation of the Witcher Tracker project, we encountered several challenges. Below is a summary of the main difficulties and how we solved them.

\begin{itemize}
    \item \textbf{Dynamic Linked List Management:}  
    I had difficulties at first managing linked lists dynamically, especially with adding, searching, and removing nodes properly. 
    \newline
    \textbf{Solution:} With the help of ChatGPT, I learned how to properly use linked lists in C. I understood the logic of pointers and implemented generic functions like \texttt{find()}, \texttt{add()}, and \texttt{removeItem()} for safe data management.

    \item \textbf{String Parsing and Input Handling:}  
    It was challenging to parse complex input commands like \texttt{Geralt trades 1 Harpy trophy for 8 Vitriol, 3 Rebis} correctly.
    \newline
    \textbf{Solution:} I used a combination of \texttt{strstr}, \texttt{strtok}, and \texttt{sscanf} functions to split the string dynamically and extract necessary values without hardcoding.

    \item \textbf{Different Sorting Requirements:}  
    Only ``What is in'' query required sorting items from most to least quantity, but other outputs had to be alphabetical.
    \newline
    \textbf{Solution:} I created a separate function \texttt{printSortedByQuantity()} and used it only in relevant parts to meet the project requirements.

    
\end{itemize}



\section{Results}

Sample Input:

\begin{verbatim}
Geralt loots 5 Rebis, 4 Vitriol
Geralt learns Black Blood potion consists of 3 Vitriol, 2 Rebis, 1 Quebrith
Geralt brews Black Blood
Total ingredient?
\end{verbatim}

Sample Output:

\begin{verbatim}
Alchemy ingredients obtained
New alchemy formula obtained: Black Blood
Alchemy item created: Black Blood
3 Rebis, 1 Vitriol
\end{verbatim}

\section{Discussion}
\textbf{Strengths:}
\begin{itemize}
    \item Clean modular code.
    \item Fully dynamic memory usage.
    \item Correct parsing of all commands.
\end{itemize}

\textbf{Limitations:}
\begin{itemize}
    
    \item Some commands require precise spacing to be parsed correctly.
\end{itemize}



\section{Conclusion}
The Witcher Tracker project successfully simulates a text-based inventory and combat assistant for Geralt. It demonstrates C programming skills such as dynamic memory allocation, string parsing, and modular code structure.

\section{References}
\begin{itemize}
\item Ritchie, Dennis M., and Kernighan, Brian W. "The C Programming Language."
\item UNIX Programming Concepts — Lecture Notes.
\end{itemize}

\section*{AI Assistants}
During the development of this project, AI tools (ChatGPT) were used for:
\begin{itemize}
    \item Debugging syntax errors.
    \item Getting suggestions for string parsing improvements.
    \item Learning linkedlist fundamentals for C. 
\end{itemize}

All code is written and implemented by the students. AI usage was limited to understanding concepts and formatting.

\end{document}